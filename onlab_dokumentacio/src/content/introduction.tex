%----------------------------------------------------------------------------
\chapter{\bevezetes}
%----------------------------------------------------------------------------

A feladat során egy olyan videó átméretező IP blokk fejlesztését kezdtem meg, amely tetszőleges méretről tetszőleges méretre tud átméretezni videófolyamot. A videó átméretezésnek több alkalmazása is lehet, pl retro konzolok képeinek upscalelése nagyobb felbontásra, de egy érdekesebb felhasználása az a mesterséges intelligenciára épülő képfeldolgozásban van, ugyanis sok ilyen modell csak bizonyos méretű képeket/videókat képes bemenetként elfogadni, és ha pl a kameránk nem ilyen videófolyamot készít, akkor az mindenképpen preprocessingre szorul.

Azért választottam ezt a feladatot, hogy jobban megismerkedjek az FPGA-kal, és a digitális tervezéssel, hiszen eddigi tanulmányaim során nem találkoztam még velük. Éppen ezért a félév első felében főleg ismerkedtem a verilog nyelvvel, és a digitális tervezéssel, közben irodalomkutatást végeztem. A félév második felében pedig elkezdtem az implementációt.

A félév során azt a célt tűztem ki, hogy képeket legyen képes a modul tetszőleges méretűre átméretezni. Azzal, hogy a bemenő képfolyam milyen módon érkezik meg az FPGA-ba, jelenleg nem foglalkoztam.

Első lépének irodalomkutatást végeztem képek átméretezéséről. Itt különböző átméretezési algoritmusokat ismertem meg, melyek közül a bilineáris transzformációval, és a polyphase transzformációval foglalkoztam részletesebben.
 
Második lépésnek megpróbáltam egy egyszerűbb bilineáris algoritmust megvalósítani, amely tetszőleges méretről tetszőleges méretűre tud képeket átméretezni.

Harmadik lépésnek létrehoztam egy sorbuffer modult testbenchel, amely a megfelelő sorokat képes eltárolni egy bejövő folyamból, és a megfelelő pixeleket tudja adagolni az átméretező algoritmusnak.

Végezetül ezeket integráltam és test benchet hoztam létre nekik.