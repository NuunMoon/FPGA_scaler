\pagenumbering{roman}
\setcounter{page}{1}

\selecthungarian

%----------------------------------------------------------------------------
% Abstract in Hungarian
%----------------------------------------------------------------------------
\chapter*{Feladatkiírás}\addcontentsline{toc}{chapter}{Kivonat}

Az átméretezés igénye számos videó feldolgozó folyamatban merül fel igényként, s mivel a videó feldolgozás FPGA-ban tipikusan jól  megvalósítható feladat, így ebben a környezetben is szükség van az ezt a feladatot ellátó egységre.

Ugyan mindkét nagy FPGA gyártó (Intel, Xilinx) rendelkezik ehhez gyári megoldással, azonban ezek az IP magok meglehetősen drágák, és utóbbi gyártó esetén ráadásul meglehetősen rosszul működik. A feladat célja egy saját, általánosan használható átméretező IP kifejlesztése.

A cél egy olyan blokk kifejlesztése, amely testzőleges felbontásról képes bármilyen másik tetszőleges felbontásra skálázni magasabb fokszámű szűrők használatával, azaz jó minőségben.



\vfill
\cleardoublepage

\selectthesislanguage

\newcounter{romanPage}
\setcounter{romanPage}{\value{page}}
\stepcounter{romanPage}